\chapter{Einleitung} 
In Anwendungen ist das Paradigma der reaktiven Programmierung nicht neu, doch hält es derzeit vermehrt Einzug in das Java-Ökosystem und etabliert sich zunehmend als Standard. Die Firma doubleSlash möchte in dieser Arbeit näher untersuchen lassen wie und an welcher Stelle die reaktive Programmierung von Vorteil ist, wann sie Sinn ergibt und wann nicht.

Da reaktive Programmierung nicht nur in höheren Programmiersprachen vertreten ist, sondern auch in Systemarchitekturen, kann doubleSlash die Erkenntnisse dieser Arbeit für künftige Projekte in Zusammenarbeit mit anderen Firmen nutzen, um schnelle und skalierbare Software zu entwickeln.

Da sich die reaktive Programmierung immer größerer Beliebtheit erfreut, muss sich doubleSlash ebenfalls mit dieser Technologie beschäftigen, um auch künftig konkurrenzfähig zu bleiben. Reaktive Systeme können langfristig die klassischen Request-Response Modelle ergänzen, da die Menge der Daten im Zeitalter von Big Data immer weiter zunimmt. 

Der Begriff der reaktiven Programmierung wird in modernen Applikationen häufig genutzt. Als problematisch erweist sich für das Unternehmen doubleSlash bislang das Fehlen von klaren Vorstellungen zu Einsatzszenarien und Möglichkeiten der reaktiven Programmierung.

Ziel dieser Arbeit ist es daher, das Thema der reaktiven Programmierung ausführlich aus einer technologischen Perspektive heraus zu diskutieren und zu evaluieren, um Entscheidungen bezüglich künftiger Einsätze in Projekten zu vereinfachen.

Die Thesis gliedert sich in neun Kapitel. Die Kapitel \hyperref[chap:reaktivitaet_in_software]{zwei}, \hyperref[chap:reaktivitaet_auf_der_systemebene]{drei} und \hyperref[chap:reaktivitaet_auf_der_anwendungsebene]{vier} befassen sich mit grundlegenden Informationen zur Reaktivität und reaktiver Programmierung.

In Kapitel \hyperref[chap:evaluierung]{fünf} werden reaktive Technologien evaluiert und gewichtet. Das anschließende Kapitel \hyperref[chap:scenario]{sechs} diskutiert Einsatzszenarien reaktiver Technologien anhand von Erfolgsgeschichten von Unternehmen wie Netflix und Paypal.

In Kapitel \hyperref[chap:concept]{sieben} und \hyperref[chap:umsetzung]{acht} wird eine reaktive Anwendung prototypisch in zwei Varianten umgesetzt und getestet: zum einen in einer reaktiven- und zum anderen in einer klassischen, blockierenden Variante.

Das Fazit in Kapitel \hyperref[chap:fazit]{neun} fasst die Ergebnisse dieser Untersuchung zusammen.