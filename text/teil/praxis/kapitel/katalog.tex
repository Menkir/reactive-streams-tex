\chapter{Kriterienkatalog}
\label{chap:kriterienkatalog}
\emph{
In diesem Kapitel geht es um eine nähere Betrachtung der Kriterien und Einsatzszenarien für Reaktive Programmierung. Dabei werden gezielt Fallstudien bekannter IT-Unternehmen untersucht, um Kriterien abzuleiten.
}

- In welchen Szenarien ist RP besser als klassische Request Response

- Defnition von klassischen Request Response als Singlethreaded/Multithreaded Anwendung die einen Request entgegen nimmt und sequentiell verarbeitet.

- Szenarien: 
	- Mutlithreaded Anwendung vs RP Anwendung 
	- Viele Daten vs wenig Daten
	- Singlethread vs single Stream
	- Wartbarkeit, Benutzerfreundlichkeit
	- Benchmark?
	
	
\section{Kriterien}
\section{Szenarien}