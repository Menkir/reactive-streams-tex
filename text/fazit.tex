\chapter{Fazit}
\label{chap:fazit}
Reaktivität ist ein Begriff der im wesentlichen in zwei Bereichen der Softwareentwicklung stark an Popularität gewonnen hat. Zum einen in reaktiven Systemen und zum anderen in reaktiver Programmierung. Während ein reaktives System eine Architektur beschreibt, die eine hochgradig antwortbereite Systeme ermöglicht, meint die reaktive Programmierung ein Programmierparadigma, mit dem asynchrone und nicht-blockierende Anwendungen entwickelt werden können. 

Die Streams sind als de facto Standard in der reaktiven Programmierung anzusehen, denn sie haben wie das reaktive Manifest ebenfalls eine Standardisierung, die Reactive Streams Specification. Anders wie Streams aus der Collection API von Java, können reactive Streams nicht-blockierend Daten mit Back Pressure verarbeiten. Das ermöglicht eine Skalierung in vertikaler Ebene, die mit den Java Streams nicht ohne weiteres nicht möglich ist. So gibt es zwar Techniken wie Futures, die sogar Teilaspekte der Reactive Streams anbieten. Allerdings können Futures keine Sequenzen von Daten übertragen und müssen den Rückgabewert über Callback-Funktionen verarbeiten. Diese Art der Verarbeitung führt häufig zur so genannten Callback-Hell. Reactive Streams lösen diese Probleme, jedoch bietet sich deren Verwendung nicht in jedem Fall an. Sind einzelne Anfragen und blockierender Code nützlicher, sollte überlegt werden, ob sich der erhöhte Lernaufwand für den Anwendungsfall lohnt. In Anwendungen, die parallel und nicht-blockierend laufen, lassen sich mit Reactive Streams verständliche und übersichtliche Algorithmen entwickeln. \\

Vorreiter auf dem Gebiet der Reactive Streams in Java ist Netflix durch die Entwicklung der RxJava Bibliothek. Diese implementiert die Reactive Streams Specification und vereinfacht die nebenläufige Programmierung durch eine deklarative API maßgeblich. In der Android-Welt erfreut sich RxJava großer Beliebtheit und gilt dort als Standardbibliothek für reaktive Programmierung. Unternehmen wie Pivotal und Lightbend bieten mit ähnlichen Bibliotheken wie etwa Reactor3 und Akka-Streams (als Teil von Akka) Alternativen. \\

Von den drei betrachteten Bibliotheken erwies sich Reactor als einsteigerfreundlichste und am besten bedien- und benutzbare Bibliothek. Dabei zeichnet sich Reactor vor allem durch eine schlanke API und eine verständliche Dokumentation aus. Allerdings ist diese Bibliothek nicht so weit verbreitet und findet vorzugsweise in Spring-Projekten Anwendung. Zu den reaktiven Bibliotheken gibt es allerdings auch Frameworks, die mehrere Technologien in sich vereinen und es sogar ermöglichen, ganze reaktive Systeme im Sinne des reaktiven Manifests zu bauen. Das beliebteste Framework unter Entwicklern ist hierbei Spring, das im Webflux Modul die Reactor-Bibliothek nutzt und in Konkurrenz zu Akka und VertX steht. Besonders Akka hat in den letzten Jahren stark an Popularität gewonnen, da es als Toolkit eine große Auswahl an Modulen aufweist und mit Scala eine Alternative zu Java bietet. \\

Die Entwicklung des reaktiven Prototyps zeigt im Wesentlichen zwei Dinge auf: Zum einen sind reaktive Programme komfortabler zu programmieren, zum anderen sind sie performanter als blockierende Programme. \\

Zusammengefasst lässt sich sagen, dass reaktive Programmierung eine moderne und sichere Art ist, nebenläufig zu programmieren. Reaktive Bibliotheken sind bereits Teil der größten Frameworks auf dem Markt und ermöglichen es, stabile reaktive Systeme zu bauen. Es ist zu erwarten, dass der Trend zur reaktiven Programmierung weiter anhält und den Standard für nebenläufige, nicht-blockierende Anwendungen setzt. \\

 Nicht zuletzt ist auch die Tatsache, dass die Java 9 Flow-Api als fester Bestandteil in das JDK aufgenommen wurde, ein deutlicher Indikator für die momentane und künftige Bedeutung der reaktiven Programmierung. 
