\newglossaryentry{Orchestrierung}% the label 
{name={Orchestrierung},% the term 
 description={Die Art und Weise, wie Dienste miteinander verknüpft sind},% brief description 
} 

\newglossaryentry{Propagation}% the label 
{name={Propagation},% the term 
 description={Verbreitung von Werten},% brief description 
} 

\newglossaryentry{Skalierbarkeit}% the label 
{name={Skalierbarkeit},% the term 
 description={Fähigkeit eines Systems zum Wachstum},% brief description 
}

\newglossaryentry{Allokation}% the label 
{name={Allokation},% the term 
 description={Reservierung von Hauptspeicher wie z. B. beim in­s­tan­zi­ie­ren von Klassen},% brief description 
}

\newglossaryentry{Programmierparadigma}% the label 
{name={Programmierparadigma},% the term 
 description={Eine Denkweise, ein Problem programmatisch zu lösen. Bekannte Beispiele sind: Objektorientierung, Imperativ, Funktional u. v. m.},% brief description 
}

\newglossaryentry{Delegation}% the label 
{name={Delegation},% the term 
 description={Die Weiterleitung der Verantwortung über Systemkomponenten im Fehlerfall},% brief description 
}

\newglossaryentry{Replikation}% the label 
{name={Replikation},% the term 
 description={Die Vervielfältigung eines Dienstes z. B. durch Docker},% brief description 
}

\newglossaryentry{Deskriptiv}% the label 
{name={Deskriptive Programmierung},% the term 
 description={Man programmiert \underline{was} gemacht werden soll, nicht \underline{wie}. Bspw. die Filter Funktion statt einer Schleife},% brief description 
}

\newglossaryentry{eventbasiert}% the label 
{name={Eventbasierte Kommunikation},% the term 
 description={Eine Kommunikationsform wobei es keinen konkret adressierbaren Empfänger gibt},% brief description 
}

\newglossaryentry{Antwortzeit}% the label 
{name={Antwortzeit},% the term 
 description={Die Zeit die eine Anwendung braucht um auf eine Anfrage zu antworten},% brief description
 plural={Antwortzeiten}
}

\newglossaryentry{Ressourcen}% the label 
{name={Ressourcen},% the term 
 description={Limitierte Softwareeinheiten wie z. B. Microservices, Threads u. v. m.},% brief description 
}

\newglossaryentry{diskret}% the label 
{name={diskret},% the term 
 description={Im Intervall voneinander getrennt},% brief description 
}

\newglossaryentry{Nebenlaeufigkeit}% the label 
{name={Nebenläufigkeit},% the term 
 description={Die wechselnde Ausführung mehrere Tasks auf einem CPU Kern/Thread},% brief description 
}

\newglossaryentry{Parallelitaet}% the label 
{name={Parallelität},% the term 
 description={Die gleichzeitige Ausführung von n Tasks auf n CPU Kernen/Threads},% brief description 
}

\newglossaryentry{Asynchronitaet}% the label 
{name={Asynchronität},% the term 
 description={Die Auslagerung einer Task, die zeitintensiv ist, auf einem anderen Thread, um blockierendes warten zu vermeiden. z. B. um die \gls{UI} bei einem Datenbank Request nicht zu blockieren},% brief description 
}

\newglossaryentry{UI}% the label 
{name={UI},% the term 
 description={User Interface (Benutzeroberfläche},% brief description 
}

\newglossaryentry{Interoperabilitaet}% the label 
{name={Interoperabilität},% the term 
 description={Zusammenarbeitsfähigkeit},% brief description 
}

\newglossaryentry{Konkatenation}% the label 
{name={Konkatenation},% the term 
 description={Verknüpfung},% brief description 
}

\newglossaryentry{Homogen}% the label 
{name={Homogen},% the term 
 description={Aus gleichartigem zusammengesetzt},% brief description 
}

\newglossaryentry{volatil}% the label 
{name={volatil},% the term 
 description={flüchtig},% brief description 
}

\newglossaryentry{Konversionsrate}% the label 
{name={Konversionsrate},% the term 
 description={Eine Rate die die Anzahl an Umwandlungen von Interessent zu Kunde beschreibt},% brief description 
}
