\chapter*{Abstract}
\setheader{Abstract}
In der vorliegenden Thesis soll die Frage beantwortet werden, worin Möglichkeiten und Szenarien der reaktiven Programmierung liegen. Dazu wird der Begriff zunächst differenziert und abgegrenzt. Aus der Sicht der Architektur beschreibt die reaktive Programmierung das Design eines Systems, das die vier Qualitäten des reaktiven Manifests umsetzt. Ein reaktives System muss somit widerstandsfähig, elastisch, und nachrichtenorientiert sein, damit es eine hohe Antwortbereitschaft garantieren kann. Um die Einsatzszenerien solcher Systeme aufzuzeigen, werden Erfolgsgeschichten von Unternehmen wie Netflix, PayPal und Verizon betrachtet. Der Einsatz reaktiver Systeme vereinfacht zum einen die Entwicklung der Projekte, zum anderen steigert er den Umsatz der Unternehmen.

Aus programmatischer Sicht bezeichnet die reaktive Programmierung ein Programmierparadigma. Die Technologien Reactive Streams, Futures und Spreadsheets sind mögliche Ausprägungen davon. Die Reactive Streams werden in dieser Arbeit näher untersucht und Implementierungen von verschiedenen Anbietern verglichen. Unter den Bibliotheken wie RxJava, Akka-Streams und Reactor stellt sich Letzteres als besonders einfach in der Handhabung heraus. Mithilfe dieser Bibliothek wurde anschließend eine reaktive Applikation entwickelt und hinsichtlich der Performance mit einer äquivalenten klassischen, synchronen Variante verglichen, wobei sich die reaktive Applikation als schneller erweist.  

Die reaktive Programmierung ist folglich ein Begriff, der sowohl in der Systemebene als auch in der Anwendungsebene vertreten ist. Unter den Möglichkeiten in der Anwendungsebene dominiert Reactive Streams, die in vielen Programmiersprachen und Implementierungen angeboten werden. Reactive Streams erlauben die Entwicklung von Anwendungen, die einen bedeutend höheren Durchsatz erzielen. Aus architektonischer Sicht ermöglicht die reaktive Programmierung Systeme zu entwickeln, die schnell, skalierbar und fehlertolerant sind und den Unternehmen eine Steigerung von Umsatz und Produktivität bescheren.


